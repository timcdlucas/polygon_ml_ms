\documentclass[11pt]{article}
\fontfamily{times}
\usepackage{graphicx}
\usepackage{amsmath}
\usepackage{geometry}
\usepackage{subcaption}

\usepackage{natbib}

%\setcitestyle{authoryear}


\geometry{verbose,tmargin=30mm,bmargin=25mm,lmargin=25mm,rmargin=25mm}
\newcommand{\templatefigures}[1]
{\noindent
\begin{minipage}{2cm}
\begin{center}
%\linespread{1}
%\begin{figure}
  \centering
	\vspace{-1cm}
  \includegraphics[height = 63px]{ISI2019only_with_date_small.pdf}\\
    %\label{matrix}
%\end{figure}
\end{center}
\end{minipage}
%
\quad
%
\begin{minipage}{12cm}
\hspace*{6.8cm}
\end{minipage}
%
\quad
%
\begin{minipage}{2cm}
\begin{center}
%\linespread{1}
\vspace{-0.9cm}
\includegraphics[scale=0.4]{imag2.jpg}\\
\end{center}

\end{minipage}

\vskip0.2cm
}


\pagestyle{empty}
\begin{document}
\templatefigures{}



\small{

\begin{center}
%The title should be centred and in bold letters. It should be informative but not too long (preferably no more than two lines).
\textbf{Model ensembles with different response variables for base and meta models: malaria disaggregation regression combining prevalence and incidence data}
\end{center}

% Model ensembles with different response variables for base and meta models:


\begin{center}
{Tim C. D. Lucas*\textsuperscript{1}; Anita Nandi\textsuperscript{1}; Michele Nguyen\textsuperscript{1}; 
Susan Rumisha\textsuperscript{1}; 
Katherine E. Battle\textsuperscript{1}; Rosalind E. Howes\textsuperscript{1}; 
Chantal Hendriks\textsuperscript{1}; Andre Python\textsuperscript{1}; Penny Hancock\textsuperscript{1}; 
Ewan Cameron\textsuperscript{1}; Pete Gething\textsuperscript{1}; Daniel~J.~Weiss\textsuperscript{1}.}\\
{1. Malaria Atlas Project, Big Data Institute, University of Oxford, Oxford, UK - timcdlucas@gmail.com}\\ 


\end{center}

\begin{center}
{\bf Abstract}
\end{center}

\setlength{\parindent}{0pt}

% This is where the abstract is placed. It should include a statement about the problem being addressed in the presentation (and paper, if submitted). Continue with a discussion of why it is important to address this problem. This may be followed by some summary information about the models and methods developed and/or used to address the problem. Conclude with a description of the key results and contributions that will be covered in the presentation (and paper).\\

Maps of infection risk are a vital tool for the elimination of malaria.
Routine surveillance data of malaria case counts, often aggregated over administrative regions, is becoming more widely available and can better measure low malaria risk than prevalence surveys.
However, aggregation of case counts over large, heterogenous areas means that these data are often underpowered for learning relationships between the environment and malaria risk.
A model that combines point surveys and aggregated surveillance data could have the benefits of both but must be able to account for the fact that these two data types are on different scales.
Here, we train multiple machine learning models on point surveys and then combine the predictions from these with a geostatistical disaggregation model that uses routine surveillance data.
We find that, in tests using data from Colombia and Madagascar, combining predictions from machine learning models out performs models using environmental covariates.

{\bf Keywords}: Spatial statistics; Ensemble; Stacking; Epidemiology.
}\\


\setlength{\parindent}{0pt}

{\bf 1. Introduction}


%We need to switch to downscaling models
%But they have issues of low power.
%While prevalence points should not be primary data source, they should still be used.
%One way is ML.

High-resolution maps of malaria risk are vital for elimination but mapping malaria in low burden countries presents new challenges as traditional mapping of prevalence from cluster-level surveys \citep{gething2011new, bhatt2017improved, gething2012long, bhatt2015effect} are often not effective due to the large sample sizes needed for accurate estimates at low prevalence rates and because of the lack of nationally representative prevalence surveys in low burden countries \citep{sturrock2016mapping, sturrock2014fine}. 
Routine surveillance data of malaria case counts, often aggregated over administrative regions defined by geographic polygons, is becoming more widely available and of better quality and recent work has focussed on methods for estimating high-resolution malaria risk from these data \citep{sturrock2014fine, wilson2017pointless, law2018variational, taylor2017continuous, li2012log}. 
However, the aggregation of cases over space means that the data may be relatively uninformative, especially if the case counts are aggregated over large or heterogenous areas. 
Given these restrictions, modelling the non-linearities known to be important in malaria mapping is difficult \cite{bhatt2017improved, bhatt2015effect}. 
A model that combines point surveys and aggregated surveillance data, and therefore leverages the strength of both, has great potential.
 
One approach for combining these data is to use prevalence point-surveys to train a suite of machine learning models, and then use predictions from these models as covariates in a model trained on polygon-level incidence data. 
This process of stacking models has proven effective in many realms however typical stacking uses a single dataset on a consistent scale \citep{sill2009feature, bhatt2017improved}. 
Here we propose training the level zero machine learning models on point-level, binomial prevalence data and stacking these models with a polygon-level, Poisson incidence model. 


{\bf 2. Methodology}

We used two data sources that reflect malaria burden; point-prevalence surveys and polygon-level, aggregated incidence data. 
We selected Colombia and Madagascar as case examples as they both have good surveillance data.
The prevalence survey data were extracted from the Malaria Atlas Project prevalence survey database using only data from 1990 onwards \citep{bhatt2015effect}.  % ceck todo
As the prevalence surveys cover different age ranges they were standardised to an age range of 2--10 using the model from \citep{smith2007standardizing}. 
For Colombia we used all points from South America (n = 522) while for Madagascar we used only Madagascan data (n = 1505).
The polygon level surveillance data were collated from various government reports. 
To account for underreporting of clinical cases due to lack of treatment seeking, missed case reports and cases that sought medical attention outside the public health systems, the methods in \citep{cibulskis2011worldwide} were used. Where species specific reports were given, these were used. For reports where no species specific, national estimates of the ratio between \emph{P. falciparum} and \emph{P. vivax} cases were used to calculate \emph{P. falciparum} only cases. To minimise temporal effects we selected, for each country, one year of surveillance data. 
We used surveillance data from 2015 for Colombia (n = 952) and data from todo for Madagascar (n = 110).

Raster surfaces of population for the years 2005, 2010 and 2015, were created using a hybrid of data from GPWv4 \citep{gpw4} and WorldPop \citep{tatem2017worldpop}, with the latter taking priority for those pixels where both had population data. 
Final population rasters were created by linear interpolation of the surrounding five-yearly rasters. 

We considered an initial suite of environmental and anthropological covariates, at a resolution of approximately $5 \times 5$ kilometres that included the annual mean and log standard deviation of land surface temperature, enhanced vegetation index, mosquito temperature suitability index, elevation, tassel cap brightness, tassel cap wetness, log accessibility to cities, log night lights and proportion of urban land cover. 
After preliminary analyses, tassel cap brightness and urban land cover were removed as they were highly correlated with other variables. 
The covariates were standardised to have a mean of zero and a standard deviation of one. 
These covariates were used for both the machine learning models and the polygon-level models.

For each region we fitted five models via caret \citep{caret}: elastic net \citep{enet}, Random Forest \citep{wright2015ranger}, projection pursuit regression \citep{friedman1981projection}, neural networks \citep{nnet} and boosted regression trees \citep{gbm}.
Our response variable was prevalence and we weighted the data by sample size (i.e. the number of people tested for malaria in each survey.
For each model we ran 5 fold cross validation to select hyperparameters using random search for Random Forest and boosted regression trees and grid search for the other models. 
%The parameters tried can be seen in Table S1.
Predictions from these models were then made across Colombia and Madagascar.
These predictions were finally inverse logit transformed so that they are on the correct scale for the top level model.

The top level model is a disaggregation regression model \citep{sturrock2014fine, wilson2017pointless, law2018variational, taylor2017continuous, li2012log}.
This model is defined by a likelihood at the level of the polygon with covariates and a spatial random field at the pixel-level. 
Values at the polygon-level are given the subscript $a$ while pixel level values are indexed with $b$.

The polygon case count data, $y_a$ is given a Poisson likelihood

$$y_a \sim \operatorname{Pois}(i_a\mathrm{pop_a})$$

where $i_a$ is the estimated polygon incidence rate and $\mathrm{pop_a}$ is the observed polygon population-at-risk. 
This polygon-level likelihood is linked to the pixel level prevealence 

$$i_a = \frac{ \sum(i_b \mathrm{pop}_b)}{\sum  \mathrm{pop}_b} $$
$$i_b = \mathrm{p2i}(p_b)$$

where $\mathrm{p2i}$ is from a model that was published previously \citep{cameron2015defining} which defines a function
$${p2i}: f\left(P\right) = 2.616P - 3.596P^2 + 1.594P^3.$$

The fact that the model passes through prevalence space allows predictions of prevalence to be made at the same time as predictions of incidence rate and that the predictions from the machine learning models are on an appropriate scale.
The linear predictor of the model, $\eta_b$, is related to prevalence by a typical logit link function.

$$p_b = \operatorname{logit}^{-1}(\eta_b)$$

The linear predictor is composed of an intercept, covariates, a spatial, Gaussian random field and an iid random effect.

$$\eta_b = \beta_0 + \beta X  + u(s, \rho, \sigma_u) + v_j(\sigma_v)$$

The Gaussian spatial effect $u(s, \rho, \sigma_u)$ has a Mat\'ern covariance function and two hyperparameters: $\rho$, the nominal range (beyond which correlation is $< 0.1$) and $\sigma_u$, the marginal standard deviation.
The iid random effect, $v_j \sim \operatorname{Norm}(0, \sigma_v)$, modelled both missing covariates and extra-Poisson sampling error

Finally, we complete the model by setting priors on the parameters $\beta_0, \beta, \rho$ and $\sigma_u$ and $\sigma_v$. 
We assigned $\rho$ and $\sigma_u$ a joint penalised complexity prior \citep{fuglstad2018constructing} such that $P(\rho < 1) = 0.00001$ and $P(\sigma_u > 1) = 0.00001$. 
We set this prior such that the random field could explain most of the range of the data if required.

We assigned $\sigma_v$ a penalised complexity prior \citep{simpson2017penalising} such that $P(\sigma_v > 0.05) = 0.0000001$. 
This was based on a comparison of the variance of Poisson random variables, with rates given by the number of cases observed, and an independently derived upper and lower bound for the case counts using the approach defined in \citep{cibulskis2011worldwide}. 
We found that an iid effect with a standard deviation of 0.05 would be able to account for the discrepancy between the assumed Poisson error and the independently derived error.
Finally, we set regularising priors on the regression coefficients $\beta_i \sim \operatorname{ Norm}(0, 0.4)$. 
The models were implemented and fitted using Template Model Builder \citep{TMB} in R \citep{R}

We compared the performance of the models with three sets of covariates.
Firstly, we used the environmental and anthropogenic covariates, centered and standardised.
Secondly, we used just the predictions from the machine learning models.
Finally we combined these two sets of covariates.

To compare the three models we used two cross validation schemes. 
In the first, polygon incidence data was randomly split into six cross-validation folds.
In the second, polygon incidence data was split spatially into three folds (via k-means clustering on the polygon centroids).
This cross-validation experiment is testing the models’ ability to make predictions far from data where the spatial random field is not informative.
Our primary performance metric was correlation between observed and predicted data.


{\bf 3. Results}

Overall, model performance was good in Madagascar under both random and spatial cross-validation (Figure~\ref{f:scatter}).
Model performance was reasonable in Colombia under random cross-validation but poor under spatial cross-validation (Figure~\ref{f:scatter}).
This indicates that the covariates alone cannot explain malaria incidence in this area.
Mapped predictions of the best performing model in Colombia look reasonable and are not unduly dominated by the random field (Figure~\ref{f:map}).

The model using only machine learning predictions as covariates was the best performing model in both countries and both cross-validation schemes (Table~\ref{t:results}).
As expected models performed better in the random cross-validation scheme than the spatial cross-validation scheme in all cases.
The difference between the covariate only model and the machine learning predictions only model was greater in the spatial cross-validation scheme than the random cross-validation.
The difference in performance between the best and worst models was always smaller than the difference between the random and spatial cross-validation schemes.

The cross-validation results for the machine learning models are shown in Table~\ref{t:mlresults}.
Also in Table~\ref{t:mlresults} is the means (across folds) of the  regression coefficients (i.e. the weights of the machine learning models in the level zero model) from the polygon-level models that used only predictions from machine learning models as covariates.
Predictive performance was similar across different models, with Random Forest performing best in Madagascar and neural networks, Random Forests and elastic net performing equally well in Colombia.
The estimated regression parameters are similar between the random and spatial cross-validation schemes.
However, the best performing machine learning models do not have the largest estimated regression coefficients as would be expected if prevalence and incidence were completely correlated.
Also of note is that some models were estimated to have a negative relationship (conditional on the inclusion of predictions from other machine learning models) with polygon incidence.

% models were quite good except spatial colombia
% which model performed best
% Random field parameters?
% coverage
% which ml model was best Vs parameters.
% parameters are negative and don't sum to one
%
% differences between models was less than difference between data.

\begin{figure}
\centering
    \begin{subfigure}[b]{0.49\textwidth}
        \includegraphics[width=\textwidth]{figs/cv1_scatter.png}
        \caption{Random cross-validation}
        \label{fig:cv1}
    \end{subfigure}
    \begin{subfigure}[b]{0.49\textwidth}
        \includegraphics[width=\textwidth]{figs/cv2_scatter.png}
        \caption{Spatial cross-validation}
        \label{fig:cv2}
    \end{subfigure}
\caption{
  Observed data against predictions for cross-validation hold-out samples on a square root transformed scale.
  a) Six-fold random cross-validation.
  b) Three-fold spatial cross-validation with folds indicated by colour.
}
\label{f:scatter}
\end{figure}



\begin{figure}
\centering
\includegraphics[trim={0 40mm 0 40mm}, width = 0.55\textwidth]{figs/col_obs_pred_map_ml.png} %\caption{Indonesia spatial crossvalidation} 
\caption{
  Left: Observed polygon data for Colombia. Grey indicates incidence of zero. Right: Out-of-sample predictions for the random cross-validation, machine learning model only model. For each cross-validation fold, predictions are made for the held out data. These predictions are all plotted which together covers the entire study area.
}
\label{f:map}
\end{figure}


\begin{table}
\caption{Pearson correlations between observed and predicted values. }
\centering
\begin{tabular}{llrrr}
Cross-validation scheme & Country &  Covariates &  ML &  Covs + ML \\
\hline 
 Random &  Colombia &  0.45 &  \textbf{0.55} &  0.54 \\
 Random &  Madagascar &  0.70 &  \textbf{0.76} &  0.75 \\
 Spatial &  Colombia &  0.05 &  \textbf{0.18} &  0.10 \\
 Spatial &  Madagascar &  0.22 &  \textbf{0.63} &  0.61 
\end{tabular}
\label{t:results}
\end{table}


\begin{table}
\caption{Machine learning model results and means of fitted parameters across $k$ model. }
\centering
\small
\begin{tabular}{l|rrr|rrr}
               &&      Madagascar &&      &                                     Colombia&    \\
Model          & ML RMSE & Random CV $\bar{\beta}_i$ & Spatial CV $\bar{\beta}_i$  & ML RMSE & Random CV $\bar{\beta}_i$  & Spatial CV $\bar{\beta}_i$  \\
\hline
nnet           & 0.113      &  0.031         &  0.025         & \textbf{0.058}& -0.250          & -0.246          \\
RF   &  \textbf{0.100}      &  0.337         &  0.350         & \textbf{0.058}&  0.782          &  0.742          \\
gbm            & 0.109      &  \textbf{0.450}&  \textbf{0.402}& 0.066         &  \textbf{0.835} &  \textbf{0.775} \\
enet           & 0.116      &  0.326         &  0.307         & \textbf{0.058}& -0.563          & -0.369          \\
ppr            & 0.110      & -0.233         & -0.204         & 0.059         &  0.210          &  0.166          \\
\end{tabular}
\label{t:mlresults}
\end{table}




{\bf 5. Conclusions}

Overall, our experiments suggest that using predictions from machine learning models trained on prevalence points is better than using environmental covariates when fitting disaggregation models of malaria incidence.
This increased performance comes despite the data being on different scales, the data being measurements of different aspects of malaria transmission and despite the imperfect model we have used to translate between the two scales.
However, the poor performance in the spatial cross-validation schemes highlights that better spatial coverage of data would improve predictions more than the improved model we have suggested.

While we have used a small suite of covariates, the fact that models such as Random Forest and elastic net can have many covariates means we could include a much larger set of environmental variables.
This is in strong contrast to previous downscaling models which had to use a very small number of covariates due to the relatively small number of polygons available and because data associated with polygons is underpowered for learning many parameters \citep{sturrock2014fine}.


% this method is an improvement
% still reliant on spatial field

% allows broad suite of covariates

% sum to one stacking
% global ml models, local experts

Our top-level polygon model is not technically stacking as we have not contrained the regression parameters to be positive nor included a sum-to-one constraint.
We did not include these constraints because the base models and the meta model are trained on response data on different scales.
However, future work could examine whether using a positive constraint on the regression parameters improves performance.

Another area of potential improvement is varying the data used to train the base level learners.
Here we only used data from the region of interest (South America for the Colombia model and Madagascar for the Madagascar model).
However, the global dataset is much larger than these subsets.
Training some base level models on local data and some on the global dataset and then combining predictions from all these models has potential to further improve model performance.


\bibliography{Malaria} 
\bibliographystyle{apalike}





\end{document}









